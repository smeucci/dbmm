\begin{tframe}{Dataset Creation}

The phase for the \textbf{dataset creation} consists in the acquisition of images exploiting the modern \textit{image search engines}.

\vspace{0.1in}

It is necessary to decide a list of \textbf{identities} that are sufficiently popular, so that for each identity there will be a high number of images, between 500 and 1000) as results of the search engines.

\vspace{0.1in}

The acquisition procedure is divided in two parts: 

\begin{itemize}
\item A collection phase.
\item A download phase.
\end{itemize}

\end{tframe}


\begin{tframe}{Collection}

The \textbf{collection} phase takes advantage of three search engines, such as Bing, Yahoo and AOL, to query and obtain images for each identity.

\vspace{0.1in}

Each HTML page returned contains the links to the images related to the query. Because every search engine has a different layout for the returned HTML pages and a different way of presenting the links in the pages, it is necessary to implement a \textbf{parser} for every search engine used.

\vspace{0.1in}

The links are then saved in the database with information about the related identity, the search engine used and the rank of the result.

\end{tframe}


\begin{tframe}{Download}


The \textbf{download} phase takes place once the database of links have been created. 

\vspace{0.1in}

The procedure is straight forward: the images are downloaded and saved in a folder related to the identity, using the links obtained in the previous step.


\end{tframe}