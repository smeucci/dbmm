\begin{tframe}{Classification}

It is necessary a \textbf{classification} phase in order to clean the dataset from the images that are not relevant to the associated identity.

\vspace{0.1in}

We consider as images the bounding boxes extracted by the face detector, in order to focus the classification to the detected faces.

\vspace{0.1in}

Taking advantage from a \textbf{pre-trained CNN} of faces from 1000 identities, we extract from the net the last layer of convolution as feature vector for each images in the dataset. For each identity, a \textbf{linear SVM} will be trained.

\vspace{0.1in}

The feature vectors were extracted by the net using the \textbf{MatConvNet} toolkit [4] and the model was trained using the \textbf{LIBSVM} library [3].

\end{tframe}


\begin{tframe}{Classification}

In order to reduce the computational complexity, each identity is trained against other K (ex. K = 5) identities randomly chosen, taking 50 images from each one.

\vspace{0.1in}

The images that are used in the training phase, were chosen by taking into account the \textbf{ranking} of the search engines, assuming that the higher the ranking the more relevant the images.

\vspace{0.1in}

It is a simple \textbf{binary} classification for each identity and so we need to train each identity separately.

\end{tframe}