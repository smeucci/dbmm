\begin{tframe}{Results}

To evaluate the performance of this procedure, \textbf{experiments} have been performed with two different datasets.

\vspace{0.1in}

One dataset (test A) contains 50 identities, such as \textbf{celebrities} and \textbf{actors}, the other (test B) contains 50 identities, such as \textbf{football players}. 

\vspace{0.1in}

The identities for the first dataset are a selection of the ones used by the pre-trained CNN that was used to extract the feature vectors for the images.

\end{tframe}


\begin{tframe}{Results}

Using the \textbf{ground-truth} created by the validation phase, it was possible to compare the results of the automatic classification.

\begin{table}[ht]
\centering % used for centering table
\begin{tabular}{c c c c c c} % centered columns (4 columns)
\hline\hline %inserts double horizontal lines
Test & TPR & TNR & FPR & FNR & Accuracy \\ [0.5ex] % inserts table
%heading
\hline % inserts single horizontal line
A & 0,993 & 0,874 & 0,126 & 0,007 & 0,973 \\
B & 0,978 & 0,839 & 0,161 & 0,022 & 0,952 \\ [1ex] % [1ex] adds vertical space
\hline %inserts single line
\end{tabular}
\label{table:nonlin} % is used to refer this table in the text
\end{table}

\vspace{0.1in}

The average number of images per identity for the \textbf{prediction} is 420 for test A and 456 for test B; the average number of images per identity after the \textbf{validation} is 412 for test A and 448 for test B.

\end{tframe}